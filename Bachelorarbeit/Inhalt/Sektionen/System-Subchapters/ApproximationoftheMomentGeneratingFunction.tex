With postulate \eqref{mpos:VollstaendigeGenerierendeFunktion} we have layed the groundwork for the practical calculation method. For it to unleash we write the operator $\Ex(\mcL_f)$ in its series expansion by assumption of convergence to 
\[
    Z_{z,R_\omega}[J] = \text{Ex}_{\mcL_f}\nsqbra{Z_{z,R_\omega}^{(0)}}[J] = \sum_{n\in\N} \frac{1}{n!}\cdot\mcL_f^n(Z_{z,R_\omega}^{(0)})[J].
\]
\input{../journal/Tage/2024-05-13/IntegralTransformation.tex}
\begin{mcor}{Perturbative Basis of the MGF with Normalization}{GeneratingFunctionBasis}
    Let $\text{Ex}(\mcL_f)$ be the development operator for the generating function $Z_{z,R_\omega}[J]$. Then its free theory argument $Z_{z,R_\omega}^{(0)}[J]$ can be written as
    \begin{align*}
        Z_{z,R_\omega}^{(0)}[J] &= \exp(
            -\frac{1}{2}\cdot\int_{\R^d} \hat J(p)\cdot G_0(p,z)\cdot\hat J(-p)\;\uplambdabar(dp)
        )\cdot N_z^{(0)}.
    \end{align*}
    Assuming $N_z^{(0)}\in\R_{\neq 0}$ the \emph{normalization factor} enforces the initial value $Z_{z,R_\omega}^{(0)}[0] = 1$.
\end{mcor}
% \input{../journal/Tage/2024-05-13/DefinitionFourierLagrange.tex}
With the free theory under Fourier transformation explicitly handled, the two interactive summands now include the two newly introduced vertices $V$ and $\mu_z$. 
The construction of $\text{Ex}$ however remains similar in the Fourier image. 
With this and \cite{mth:vogel} we therefore find:\footnote{The integration domain of the second summand is given by the hyperplane introduced due to the limitation $\drho_{R_\omega}(-(p_1 + p_2 + p_3))$. We assume this to be convered in the measure $\overline\mu_{\overline{\text{III}}}$ and do not discuss it explicitly, since for the one loop order the corresponding Lagrangian integral $\mcL_{f,\overline{\text{III}},\mcF}$ vanishes.}
\begin{align*}
    \mcL_{f,\overline{\text{II}},\mcF} &:= \int_{(\R^d)^2}\mu_z(p)\cdot\bbbra{\frac{\delta}{\delta J(-p_1)}\circ\frac{\delta}{\delta J(-p_1-p_2)}}(\cdot)\;\overline\mu_{\overline{\text{II}}}(dp), \\
    \mcL_{f,\overline{\text{III}},\mcF} &:= \underset{\drho_{134}(\R^d)}{\iiint} V_z(p_3,-p_2)\cdot\bbbra{\frac{\delta}{\delta J(-p_1)}\circ \frac{\delta}{\delta J(-p_3)}}(\cdot)\;\overline\mu_{\overline{\text{III}}}(dp).
\end{align*}
Notice how we can now connect the generating function to our main object of interest, the resolvent. Since expressed with extrapoladed gaussian density, the resolvent involves a factor $\Phi(x)\cdot\Phi(y)$ combined with the density $\Phi\mapsto \exp(S_{z,R_\omega}[\Phi])$, we can extend the integral using the same field shifting technique to 
\begin{multline}
    \int_{\mcF_{d,N}}\Phi(x)\cdot\Phi(y)\cdot \exp((\text{S}_{z,R_\omega}\Phi)[J])\;\uplambda(d\Phi) \\
    = \int_{\mcF_{d,N}}\bbbra{\frac{\delta}{\delta J(-x)}\circ\frac{\delta}{\delta J(-y)}}\cdot{\exp}\bbra{(\text{S}_{z,R_\omega}\Phi)[J]}\;\uplambda(d\Phi).
    \label{eq:GeneratedSecondMomentIntegral}
\end{multline}
The exponential now exactly represents the generating function $Z_{z,R_\omega}[J]$, such that the second moment against gaussian measure can be derived using two functional derivatives.


\subsubchapter{Theoretical Application of the Propagator Representation}
\input{../journal/Tage/2024-05-13/Application-I.tex}
\begin{mlem}{Vanishing third derivative of $Z_{z,R_\omega}^{(0)}$}{VanishingThirdDerivative}
    For $p\in(\R^d)^3$ the composed directional derivatives of $Z_{z,R_\omega}^{(0)}[J]$ vanish, i.e. $\frac{\delta}{\delta \hat J(p_1)}\frac{\delta}{\delta \hat J(p_2)}\frac{\delta}{\delta \hat J(p_3)}Z_{z,R_\omega}^{(0)}[J] = 0$.
\end{mlem}
\begin{proof}
    Let $Z_{z,R_\omega}^{(0)}[J]$ be defined as in \ref{mcor:GeneratingFunctionBasis}, $\Delta_p:=\frac{\delta}{\delta \hat J(p_1)}\frac{\delta}{\delta \hat J(p_2)}\frac{\delta}{\delta \hat J(p_3)}$ and $\frac{\delta}{\delta \hat J(\xi)}Z_{z,R_\omega}^{(0)}[J] = G_0(\xi,z)\cdot \hat J(-\xi)\cdot Z_{z,R_\omega}^{(0)}[J]$, then we can directly calculate the first directional derivative as
    \[
        \Delta_pZ_{z,R_\omega}^{(0)}[J] = \frac{\delta}{\delta \hat J(p_1)}\frac{\delta}{\delta \hat J(p_2)} N_z^{(0)}\cdot \bbra{
            G_0(p_3,z)\cdot \hat J(-p_3)
        }\cdot Z_{z,R_\omega}^{(0)}[J].
    \]
    Following the product rule and using the invariance of $N_z^{(0)}$ in respect to $J$ we obtain
    \begin{multline*}
        \Delta_pZ_{z,R_\omega}^{(0)}[J] = N_z^{(0)}\cdot\frac{\delta}{\delta \hat J(p_1)}\bigl(
            G_0(p_3,z)\cdot \delta(-p_1-p_2)\cdot Z_{z,R_\omega}^{(0)}[J] \\
            + G_0(p_3,z)\cdot \hat J(-p_3)\cdot G_0(p_2,z)\cdot \hat J(-p_2)\cdot Z_{z,R_\omega}^{(0)}[J]
        \bigr).
    \end{multline*}
    The last directional derivative gives us for the first summand
    \[
        G_0(p_3,z)\cdot \delta(-p_1-p_2)\cdot G_0(p_1,z)\cdot \hat J(-p_1)\cdot Z_{z,R_\omega}^{(0)}[J],
    \]
    which vanishes for $J = 0$. The second summand again uses the product to yield three individual terms, which again all vanish for $J = 0$. 
\end{proof}
\noindent With the lemma we can safely assume that the first approximation order vanishes for $J = 0$ and therefore the second order is the first non-vanishing term in our approximation. This behaviour in uneven moments can be observed in a generalizing manner: For so called \emph{centeralized gaussian vectors} $X$ the following theorem of Isserlis/Wick holds.
\begin{msat}{Isserlis'/Wick's Theorem}{IsserlisTheorem}
    Let $X:[N]\to(\Omega\to\R^d)$ be a random variable vector on $(\Omega,\mcA,\mbbP)$ with $\mbbE(X_i) = 0$ for all $i\in[N]$. Let furthermore be $\mbbP\gg\uplambda^d$ with normal distribution $f_{\Sigma}$, then it holds that 
    \[
        \mbbE\Bbbra{
            \prod_{i\in[N]}X_i
        } = \sum_{p\in P_N^2}\prod_{(i,j)\in p}\mbbE(X_i\cdot X_j),
    \]
    in which $P_N:=\{M\subset\mcP([N]):\#M = 2\;\&\;\bigcup M = [N]\}$ is the set of all pair partitions to $[N]$.
\end{msat}
\begin{proof}
    The proof uses the concept of generating functions and can be found in \cite{skript:GaussCalc}.
    % Define $t\mapsto \exp(t\cdot X)$ ... 
\end{proof}
From the theorem it directly follows that for $N\in\textit{uneven}$ we have an empty sum over $P_N^2$ and therefore the expectation value vanishes. We have to consider that Isserlis' theorem is only applicable for \emph{normally distributed} random variables!



\subsubchapter{Visualization: Calculation of one One-Loop Correction Integral}
\label{subsubchapter:VisualizationOneLoop}
\input{../journal/Tage/2024-05-13/OneLoopCalc.tex}
