Having established this basic mathematical framework, we can now move on to describing the \emph{physics} of our system. The ERM model stated in \cite{paper:Grigera_2011,mth:vogel,paper:2405.06537v1,PhysRevLett.130.236101} and other is used to describe a classical physical system of $N\in\N$ interacting particles in a $d$-dimensional space. A visualization was done in the chapter's introduction, see figure \ref{fig:GraphRepresentation}. This space where the particles individually are \emph{located} is from now on referred to as $\Gamma\subset\R^d$ with some technical assumptions on its structure. A summary of all particles positions is encoded in the \emph{phase space} $V_{d,N}$ which is given by the cartesian product of $N$ copies of $\Gamma$. 
\begin{mdef}{Modelling Space}{ModellingSpace}
    % Wir nennen $\Gamma\subset\R^d$ für $d\in\N_{\geq 2}$ kompakt in unserem Setting den \emph{Modellraum} der Teilchen. Wir verstehen dann unter $V_{d,N}$ den \emph{Ortsraum}, als ein beliebiges Element aus $\{V\in (\R^d)^N:\forall i\in[N]:V_i = \Gamma\}$.
    We call an $\R^d$-open, relatively compact subspace $\Gamma\relcom\R^d$ for $d\in\N$ the \emph{modelling space} of the particles. We then understand $V_{d,N}:=\Gamma^N$ to be the \emph{phase space}.\footnote{Implicitly we also define for technicality an isomorphism $\iota:(\R^d)^N\to\R^{d\cdot N}$ by $\iota(v) = (v_{1,1},\ldots,v_{1,d},\ldots,v_{N,1},\ldots,v_{N,d})$. To not overcomplicate the notation we will not further distinguish between $(\R^d)^N$ and $\R^{d\cdot N}$.}
\end{mdef}
Our general idea now is to assume an at least three times (see T.\ref{msat:TaylorTheorem}) continuously differentiable potential function $U:V_{d,N}\to \R$ describing the system state, i.e. the potential energy of $N\in\N$ particles numbered $i\in[N]$ placed at positions $r_i\in\Gamma$. For later purposes of approximation we would now like to only consider positional arguments which fulfill $U$-critical conditions, i.e. physically speaking \emph{equilibrium} states of the system. These also do exist for $U$ since we made according assumptions on $\Gamma\relcom\R^d$.
\begin{mdef}{Critical Distribution of Positions}{KritOrtVert}
    % Für $(\Omega,\mcA,\mbbP)$ Maß- und $(V_N,\mcB)$ Messraum sei $r\in(\Omega\to V_N)$ eine Ortsverteilung. Genau dann nennen wir $r$ kritisch bezüglich $U_\#$, wenn für alle $\omega\in\Omega$ gilt $r(\omega)\in\text{kritPkt}(U)$, also $dU(r(\omega)) = 0$.
    We call $r_*\in V_{d,N}$ a critical distribution of positions with respect to $U$ if $dU(r_*) = 0$. We denote such to be in $\text{critP}(U)$.
\end{mdef}
From this mathematical definition we can already deduce physical properties, since being in a critical state physically means that \emph{movement} is (even instably) \emph{restricted}. One could now wonder what happens if we dislocate the particles from their equilibrium positions everso slightly. Intuitively, if we have an instable equilibrium, i.e. mathematically a saddle point, we would expect the particles to move to lower states of potential energy, i.e. minima of $U$. But even in stable equilibria, our physical intuition expects a \emph{restoring force} to act on the particles, which would bring them back to their equilibrium positions. Mathematically we can analyse this behaviour by looking at the first orders of the \emph{Taylor series} of $U$. But if we want our positions to be and remain within a restricted volume $\Gamma\subseteq\R^d$, we need to first artificially restrict allowed variations of $r_*\in\text{critP}(U)$ to \emph{Taylor fluctuations}, i.e. elements of $\text{TF}(\Gamma,r_*)$.
\begin{mdef}{Taylorfluctuations}{Taylorfluktuation}
    % Sei $r_*\in\Gamma^N$ ein Ortspunkt, dann definieren wir $\phi\in\R^n$ als \emph{Taylorfluktuation} um $r_*$, falls $\phi_i\in B_{\text{dist}(r_i,\Gamma)}(0_{\R^n})$ für $i\in[N]$ gilt.
    Let $r\in V_{d,N}$, then we define $\phi\in(\R^d)^N$ as \emph{Taylorfluctuation} around $r$, if\footnote{We denote $\text{dist}(r_i,\Gamma):=\inf_{x\in\partial\Gamma}(\dabs{x - r_i}{2})$ as the minimal distance from any point $r_i$ to the boundary of $\Gamma$.} $\phi_i\in B_{\text{dist}(r_i,\Gamma)}(0_{\R^n})$ for $i\in[N]$. We denote such to be in $\text{TF}(\Gamma,r)$. 
\end{mdef}
Since for Taylor approximation from theorem \ref{msat:TaylorTheorem} we assume our displacement to be small for the remaining summand $R_{p+1}(f,r_*,h)$ to be equally small, we always find allowed $\phi\in\text{TF}(\Gamma,r_*)$ for any $r_*\in\text{critP}(U)$ because of the openness of $\Gamma$. If we perform a large volume limit on $V_{d,N}$, we will be able to neglect the restriction to $\text{TF}(\Gamma,r_*)$ and consider all of $\R^d$ as our displacement space. Another approach is a global shift of the modelling space's origin to $r_*$, by which the Taylor fluctuations can be drawn from a constant set $\text{TF}_{\Gamma,0} =: \text{TF}_{\Gamma,N}$, which can be pictured as shown in figure \ref{fig:ERMmapping}.
\begin{figure}[H]
    \centering
    \begin{tikzpicture}
        \draw[fill=black] (0,0) circle (0.05);
        \draw[] (-3,1.5) -- (3,1.5) -- (3,-1.5) -- (-3,-1.5) -- (-3,1.5);

        \draw[fill=black] (-2,1) circle (0.05);
        \draw[->] (0,0) -- (-2,1);

        \draw[dotted] ({-3-2},{1.5+1}) -- ({3-2},{1.5+1}) -- ({3-2},{-1.5+1}) -- ({-3-2},{-1.5+1}) -- ({-3-2},{1.5+1});

        
        \draw[fill=black] (1,0) circle (0.05);
        \draw[->] (0,0) -- (1,0);

        \draw[dotted] ({-3+1},{1.5+0}) -- ({3+1},{1.5+0}) -- ({3+1},{-1.5+0}) -- ({-3+1},{-1.5+0}) -- ({-3+1},{1.5+0});

        
        \draw[fill=black] (-1,-0.5) circle (0.05);
        \draw[->] (0,0) -- (-1,-0.5);
        
        \draw[dotted] ({-3-1},{1.5-0.5}) -- ({3-1},{1.5-0.5}) -- ({3-1},{-1.5-0.5}) -- ({-3-1},{-1.5-0.5}) -- ({-3-1},{1.5-0.5});
    \end{tikzpicture}
    \caption{Mapping of $\text{critP}(U)$ to $\Gamma_2$ in the ERM model.}
    \label{fig:ERMmapping}
\end{figure}
Within the ERM model we are not interested in the particles moments, such that the hamiltonian of the system is fully given by the potential energy $U$.
\begin{mdef}{The Hamiltonian}{Hamiltonian}
    Let $U\in C^2(V_{d,N})$ for $\Gamma\relcom\R^d$ and $V_{d,N}:=\Gamma^N$ be a potential energy function of a system of $N\in\N$ particles. For $r\in V_{d,N}$ and $\phi\in\text{TF}(V_{d,N},r)$ we define the corresponding $H(r,\phi):=U(r + \phi)$. Its approximation will be denoted by $H^{(n)}(r,\phi):=\sum_{k=0}^n\frac{1}{k!} D_\phi^kU(r)^k$.
    
    \footnote{We denote the differential operator $D_\phi^k$ as the $k$-th derivative of $U$ with respect to its argument $r$ in direction $\phi$. At first order we therefore have $D_\phi U(r) = \dv{t}U(r+t\phi)\big|_{t=0}$.} 
\end{mdef}
The Potential $U$ (now equivalent to its hamiltonian $H$) can now be approximated at $r_*\in \text{critP}(U)$ to second order with displacement $\phi\in\text{TF}(V_{d,N},r_*)$ by 
\[
    H^{(2)}(r_*,\phi) = U(r_*) + \ubra{\dv{t}U(r_*+t\cdot\phi)\big|_{t=0}}{=0} + \frac{1}{2}\cdot\scpr{\text{H}U(r_*)\cdot\phi}{\phi},
\]
where $\text{H}U(r_*)$ denotes the hessian of $U$ at $r_*$. This now as initially stated gives an integral of the form
\[
    \int_{V_{d,N}}\int_{\text{TF}(V_{d,N},r_*)}\exp\bbra{-\beta\cdot U(r_*)}\cdot\exp(-\beta\cdot\frac{1}{2}\cdot\scpr{\text{H}U(r_*)\cdot\phi}{\phi})\;\uplambda(d\phi)\;\uplambda(dr_*).
\]
Within the following pages representing the current standpoint of research one assumed $\exp(-\beta\cdot U(r_*))$ to be constantly given by the a priori density $1/\abs{V_{d,N}}$ with $\abs{V_{d,N}}:=\uplambda(V_{d,N})$ and therefore of no greater interest. With $\text{TF}(V_{d,N},r_*)$ also assumed to be \emph{independent}\footnote{The error made by this assumption is not to be discussed in this thesis.} of $r_*$ (see above), such that it is set to a fixed set $\text{TF}_{\Gamma,N}\subset(\R^d)^N$. This shifted the startpoint from $\int 1\;\mbbP$ effectively approximating to
\begin{multline*}
    \int_{V_{d,N}}\int_{\text{TF}(V_{d,N},r_*)}\exp(-\beta\cdot\frac{1}{2}\cdot\scpr{\text{H}U(r)\cdot\phi}{\phi})\;\uplambda(d\phi)\;\Bbra{\frac{1}{\abs{V_{d,N}}}\cdot\uplambda}(dr) \\
    \approx \int_{V_{d,N}\times\text{TF}_{\Gamma,N}}\ubra{\frac{1}{\abs{V_{d,N}}}\cdot\exp(-\frac{\beta}{2}\cdot\scpr{\text{H}U(r)\cdot\phi}{\phi})}{=\,\exp(-\beta\cdot H_*(r,\phi))}\;\uplambda(d(r_*,\phi)),
\end{multline*}
defining an a priori Hamiltonian $H_*$, in which the distribution of the shifting vector $\phi$ is now much more prominent. With growing progress in this thesis investigation it will become clear that the original exponential term is indeed of interest, which will lead to a corrected hamiltonian $H$. But first we need to discuss current integrand at hand. A regard to notation: In referenced literature these two integrals are noted using two different symbols. For averaging over the critical position domain it is established to use an overline \enquote{\,$\overline{\phantom{aaa}}$\,}, while for the translational domain of $\phi$ the symbol \enquote{$\langle \ldots\rangle$} is used, see for example \cite{paper:Grigera_2011,mth:vogel}. \\

We now need to motivate the use and definition of ERM for our system. This can be done by assuming \emph{pair interactions} between the particles. This is a common assumption in physics, since it is often the case that particle potentials can be written as a sum of pair potentials. Therefore with $r\in V_{d,N}$ we have the representation
\[
    U(r) = \frac{1}{2}\cdot\sum_{i,j\in[N]}U(r_i-r_j).
\]
Performing two differentiations on $U$ with respect to $r$ we get for $[N]^2\ni(i,j)\mapsto d_{i,j}:r\mapsto r_i-r_j$ the form $\bbra{\dv{r}}^2 (U\circ d_{i,j}) = U''(d_{i,j})\cdot d'_{i,j}^2 + U'(d_{i,j})\cdot d''_{i,j}$. For $d'_{i,j}(r) = e_i + e_j$ and $d''_{i,j}(r) = 0$ we find using $d'_{i,j}(r)^\top\cdot g'_{i,j}(r) = 2$ the second derivative as
\begin{align}
    U''(r) = \frac{1}{2}\cdot\sum_{(i,j)\in[N]^2}\bbbra{\dv{r}}^2(U\circ d_{i,j}) = \frac{1}{2}\cdot 2\cdot\sum_{i,j\in[N]}U''(r_i-r_j).\label{eq:SecondDerivativePairInteraction}
\end{align}
This physically represents exactly the idea of a feathered potential, since for an energy $E(x) = k\cdot x^2/2$ we have $E''(x) = k$. Therefore we can treat $U''$ as a spring function that yields a $k$ dependent on the distance of two particles. If we now neglect the directions, i.e. making $U''$ a radial function, we can implement a pair interaction function $f:\R^d\to\R$ that is a radial function an constitutes the hessian of $U$ with the following properties.
\begin{mdef}{Pair Interaction Function (PIF)}{PairInteractionFunction}
    We call $f\in\mcS(\R^d)$ a \emph{pair interaction function} if $\exists f_{\abs{}}\in\R\to\R$ such that $f = f_{\abs{}}\circ\dabs{\cdot}{2}$, i.e. $f$ is radially symmetric, and $f \geq 0$. \color{black} Furthermore for $x\in \partial B_r(0)$ with $r\to\infty$ we require $f(x) = 0$. % \color{red} SHOULD WE ALSO REQUIRE THAT $f \approx p^2$ FOR $p\in\partial B_r(0)$ if $r\to 0$? \color{black}
    We denote such functions to be in $\textit{Spr}\,(\R^d)$.
\end{mdef}

Physically speaking, by introducing a radial symmetry with $f = f_{\abs{}}\circ\dabs{\cdot}{2}$, we want the particles to interact equally with each other without any preference of direction. The second condition of $f$ being a Schwartz function is a technical condition to ensure fourier transformability. Comparing results to $H^{(2)}$ now gives us 
\[
    H(r_0,\phi) \approx \frac{1}{2}\cdot\sum_{i,j\in[N]}U(r_i - r_j) + \frac{1}{4}\cdot\sum_{i,j\in[N]}f_{\abs{}}(\dabs{r_i-r_j}{2})\cdot\phi_i\cdot\phi_j,
\]
with $r_0\in\text{critP}(U)$ and $\phi\in\text{TF}(V_{d,N},r_0)$. This is also a good moment to discuss units in our model. It would be natural to assume $r$ having the units of lenght, i.e. $\si{\metre}$, as well as $\phi$ which is directly added to $r$. But since a mathematical theory of units and dimensional analysis is way beyond the scope\footnote{See for example the \href{https://en.wikipedia.org/wiki/Buckingham\_\%CF\%80\_theorem}{\textit{Buckingham $\pi$ theorem}}, the paper \href{https://arxiv.org/pdf/2108.08703}{\textit{Dimensioned Algebra: the mathematics of physical quantities}} by Carlos Zapata-Carratala or others.} of this thesis, we will assume all occuring physical quantities to be dimensionless, which can be achieved by multiplication with a normalized inverse quantity. In practice this means $r = r\cdot\sigma_{\si{\metre}}^{-1}$ as well as $\phi = \phi\cdot\sigma_{\si{\metre}}^{-1}$ and therefore $[r] = 1 = [\phi] = [U] = [G]$ \cite{mth:vogel}. 

Having introduced the pair interaction function $f$ we can now take a look at our aimed approximation of the hessian of $U$ at $r_*\in\text{critP}(U)$. The core of the repulsive ERM theory given in \cite{paper:Grigera_2011,mth:vogel} is a \emph{laplacian} matrix $L\in\R^{N\times N}$, which is strongly related to graph theory. Without going too much into detail, using D.\ref{mdef:LaplacianMatrix} the decomposition $L = B\cdot B^\top = D - W$ with $B$ being the incidence matrix (see D.\ref{mdef:IncidenceMatrix}), $D$ the degree matrix (see D.\ref{mdef:DegreeAndDegreeMatrix}) and $W$ the weighted adjacency matrix (see D.\ref{mdef:WeightMatrix}) of a graph (see D.\ref{mdef:Graph}) can be stated \cite{bookchapter:GraphsAndLaplacians}. The first matrix $B$ encodes the \emph{bonds} between the particles as their signed square roots $\pm\sqrt{L_{i,j}}$, respectively for \emph{incoming} and \emph{outgoing} edges. This matrix will be holding the pair interaction function as $\pm\sqrt{f}$ in its entries. 
From it we can deduce $\scpr{L\cdot x}{x} = \scpr{B^\top x}{B^\top x} = \dabs{B^\top\cdot x}{2}^2 \geq 0$ with $B$ symmetric for all $x\in\R^{d\cdot N}$, which is the property of positive semi-definiteness. The second property\footnote{Sometimes one writes $\sum_{i\sim j}$ for nodes $(v_1,\ldots,v_N)$. This means summing only over all \emph{adjacent} nodes. Because in our system every particle interacts with every other particle, we can write $\sum_{j\in[N]}$ instead.} $i\mapsto \sum_{j\in[d\cdot N]}L_{i,j} = 0$, i.e. the rows of $L$ summing up to zero, translates into $1_{\R^{d\cdot N}} = (1,\ldots,1)$ being an eigenvector to the eigenvalue $0$. % \color{red} USE? \color{black}.

This directly corresponds to the phyical picture: The nodes of the graph are the particles, the edges are the interactions between them and the weights define the strength of the interaction. 

The euclidean part in ERM comes from the constraint of our target space $V_{d,N}$ having euclidean structure, which is straightforwardly given by the relation $V_{d,N} \subseteq \R^{d\cdot N}$ and the induced euclidean norm\footnote{The norm can indeed be freely chosen since we have an equivalence theorem for all $p$ norms build by $\dabs{x}{p}:=\Bbra{\sum_{i\in[d]}\abs{x_i}^p}^{1/p}$ for $p\in\N$ and $x\in\R^d$.} $\dabs{\cdot}{2} := \sqrt{\scpr{\cdot}{\cdot}}$ that is already part of Definition \ref{mdef:PairInteractionFunction}. This intuitively corresponds to our physical understanding of three dimensional space. With this we can now introduce the heart of the ERM model, the laplacian matrix $\tilde U(f,r_*)$.
\begin{mpos}{Coupling Representation}{CouplingRepresentation}
    %Sei $U\in C^3(\Gamma^N)$ eine Potentialfunktion. Dann nehmen wir die Existenz einer Paarwechselwirkungsfunktion $f:\R^3\to\R$ an, sodass die Hessematrix $HU(r_*)$ in $r_*\in V_N$ durch $f$ ausgedrückt werden kann. Dies passiert durch die Form $\scpr{HU(r_*)\cdot\phi}{\phi} \approx \phi^T\tilde U(f,r)\cdot\phi$. Dabei hat $\tilde U(f,r)$ die Form
    Let $U\in C^3\bbra{(\R^d)^N}$ describe the pair potential function. For\footnote{As in $d_{i,j}(r) := r_i - r_j$ stating the difference vector of $r_i$ and $r_j$ following equation \eqref{eq:SecondDerivativePairInteraction}.} $(i,j)\mapsto f_{i,j} = U''\circ g_{i,j}$ it holds\footnote{Technically speaking for multiplication we need to use an isomorphism $(\R^d)^N\to\R^{d\cdot N}$ to make the matrix multiplication work. For this we propose a projection $\pi$ of the form $\bbra{(.,.,.,.),...}\mapsto (.,.,.,.,...)$. Then we would have $\scpr{\text{H}U(r_*)\cdot\phi}{\phi} = \langle\tilde U(f,r)\cdot\pi(\phi),\pi(\phi)\rangle$.} $\scpr{\text{H}U(r)\cdot\phi}{\phi} = \langle\tilde U(f,r)\cdot\phi,\phi\rangle$ for $r\in V_{d,N}$ and $\phi\in\text{TF}(\Gamma,r)$ with $\tilde U(f,r)$ having the general form
    % Then we assume $\exists f\in\textit{Spr}\,(\R^d)$ such that for $r\in V_{d,N}$ and $\phi\in\text{TF}(\Gamma,r)$ it holds $\scpr{\text{H}U(r)\cdot\phi}{\phi} = \langle\tilde U(f,r)\cdot\phi,\phi\rangle$ with $\tilde U(f,r)$ having the form
    \[
        \tilde U(f,r_*) := \begin{pmatrix}
            \Sigma(f,1)\cdot I_d & -\tilde f_{1,2}\cdot I_d & \cdots & -\tilde f_{1,N}\cdot I_d\\ 
            -\tilde f_{2,1}\cdot I_d & \Sigma(f,2)\cdot I_d & \cdots & -\tilde f_{2,N}\cdot I_d\\
            \vdots & \vdots & \ddots & \vdots\\
            -\tilde f_{N,1}\cdot I_d & -\tilde f_{N,2}\cdot I_d & \cdots & \Sigma(f,N)\cdot I_d
        \end{pmatrix}\in\R^{d\cdot N\times d\cdot N}
    \]
    We denoted $f_{i,j} := f(r_i-r_j)$ for $i,j\in[N]^2$ and $\Sigma(f,i) := \sum_{j\in[N]\setminus\{i\}}f_{i,j}$.
\end{mpos}
% \begin{proof}
%     Wir zeigen $\tilde U(f,r) \in (\R^{3\times 3})^{N\times N}$. Hierzu beachte die Hessematrix $HU(r) := \fdef{d^2U(r)(e_i)(e_j)}{(i,j)\in[N]^2}$ für $i,j\in\{1,\ldots,3N\}$ und Basisvektoren $e_i\in \Gamma^N$. Diese haben stets die kanonische Form $e_i = (0_{\R^3},\lcdots,\mbbEins_{\sigma(i)},\ldots,0_{\R^3})$ für eine Abzählung $\sigma:\{1,\ldots,3N\}\to $
% \end{proof}
Note that if we would introduce the PIF $f$ with the property $f(0) = 0$ we can simply take the sum over all matrix columns, i.e. $\sum_{j\in[N]}f_{i,j}$. Because $\tilde U(f,r_*)$ is constructed of block matrices, we physically only use the norm of $\phi$, i.e. only its \emph{length}, not its \emph{direction}. This is a reduction of complexity, as the matrix in general holds non-diagonal blocks. Therefore a smaller matrix $M(f,r_*)$ can be defined as $M(f,r_*)_{i,j}$ being the diagonal entry of the $i,j$-th block of $\tilde U(f,r_*)$. We do not notationally differentiate the matrix $\tilde U(f,r_*)\in\R^{N\times N}$ where $\tilde U(f,r_*)_{i,j}$ is the diagonal entry of the $i,j$-th block from the general matrix $\tilde U(f,r_*)\in\R^{d\cdot N\times d\cdot N}$.\\

To introduce the randomness in euclidean \emph{random} matrices we now only need to exchange the fixed positions $r_*\in V_{d,N}$ by random variables coming from an abstract probability space $(\Omega,\Sigma,\mbbP)$ mapping to $(V_{d,N},\mathcal{B}(V_{d,N}))$. This makes our matrix representation also random, i.e. $\Omega\ni\omega\mapsto \tilde U(f,R_\omega)$ becoming a \emph{random matrix}, where $R:\Omega\to V_{d,N}$ is said random variable of positions. Doing so we find for every $\omega\in\Omega$ a initial distribution of positions in our space $\Gamma$, on which we later can calculate the expectency via the pushforwad measure $\mbbP_R:=\mbbP\circ R^{-1}$. 
\begin{mdef}{Position Distribution}{PositionDistribution}
    Let $(\Omega,\mcA,\mbbP)$ be a probability space and $V_{d,N}$ a position space. A \emph{position distribution} then is a random variable $X:\Omega\to V_{d,N}$ for $N\in\N$ particles. 
\end{mdef}