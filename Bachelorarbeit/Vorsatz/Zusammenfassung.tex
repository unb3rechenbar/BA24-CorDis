In dieser Arbeit wird der Einfluss korrelierter Unordnung in Partikelpositionen im Kontext des euklidischen Zufallsmatrixmodells diskutiert. Dabei handelt es sich um ein isotropisches System von $N$ Teilchen in einem $d$-dimensionalen Raum, dessen Dynamik durch eine Federfunktion der Paarwechselwirkung gegeben und zeitlich athermisch eingeforen ist. Daraus resultiert eine Zufallsmatrix, deren spektrale Eigenschaften unter Verwendung eines feldtheoretischen Ansatzes diskutiert werden. Nach Einführung des Dyson-Formalismus und Herleitung der Feynman-Regeln wird der statische Strukturfaktor $S_*$ eingeführt, welcher auf der Fouriertransformation der radialsymmetrischen Verteilungsdichte der Teilchenpositionen $g_0 - 1$ beruht. Unter Verwendung der selbstkonsistenten Born Näherung wird die Selbstenergie im Einklang dessen korrigiert und die entsprechende Dyson-Gleichung hergeleitet. Die Feynman-Regeln werden ebenfalls korrigiert. 

Durch die numerische Lösung der Dyson Gleichung im Beispiel eines dreidimensionalen Raums und einer Stufenfunktion als Federfunktion wird der Einfluss des statischen Strukturfaktors auf die Schallgeschwindigkeit diskutiert. Dabei wird gezeigt, daß zwar der Strukturfaktor für alle getesteten Dichten stabil im positiven Bereich liegt, jedoch keinen signifikanten Einfluss auf die Schallgeschwindigkeit ausübt. Die Dispersionsrelation wird für alle Dichten als nahezu unverändert gezeigt. Ebenfalls wird die physikalische Anwendbarkeit im Hinblick auf die Positivität der Dispersionsrelation diskutiert. Abschließend wird das Modell mit einer gaußschen Radialverteilung verglichen und das Ergebnis verifiziert.