\begin{mdef}{First Functional Variation}{ErsteFunktionalvariation}
    Sei $(V,\tau_V)$ ein topologischer Vektorraum und $F:D(F)\to \mbbK$ für $\mbbK\in\{\R,\C\}$ ein Funktional. Existiert dann ein lineares Funktional $A:V\to\mbbK$, sodaß für $f\in D(F)$ der Grenzwert
    \[
        \lim_{g\to f}\frac{\dabs{F[g] - F[f] - A(g-f)}{}}{\dabs{g - f}{}} = 0
    \]
    existiert, so wird $A$ als \emph{erste Funktionalvariation} von $F$ bezeichnet und mit $\delta F[f]$ notiert.
\end{mdef}